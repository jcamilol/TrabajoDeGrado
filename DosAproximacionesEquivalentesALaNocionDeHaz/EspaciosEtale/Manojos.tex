\begin{Def}[Manojo]
    Un manojo sobre un espacio topológico $X$ es una pareja $\left( Y, p\right)$ con $Y$ un espacio topológico y $p:Y\to X$ una función continua.
\end{Def}
Según el contexto, es frecuente referirnos al manojo $\left( Y,p\right)$ simplemente por $p$. La colección de todos los manojos sobre un espacio topológico tiene estructura de categoría:
\begin{Def}[Categoría $\textbf{Top}/X$]
   La categoría $\Top /X$ (léase ``categoría Top sobre X") o $\Bund (X)$ tiene por objetos todos los manojos sobre $X$. Dados $\left( Y,p\right),\left( Y',q\right)\in \Top/X$, $f$ es una flecha $\left( Y,p\right)\to\left( Y',q\right)$ en $\Top/X$ si $f:Y\to Y'$ es una función continua y $q\circ f=p$.
   \begin{center}
    % https://tikzcd.yichuanshen.de/#N4Igdg9gJgpgziAXAbVABwnAlgFyxMJZABgBpiBdUkANwEMAbAVxiRAE0QBfU9TXfIRQBGclVqMWbdgHJuvEBmx4CRUcPH1mrRCAAa3cTCgBzeEVAAzAE4QAtkjIgcEJKIna2l+VdsPETi5IAEzUWlK6aCDUDHQARjAMAAr8KkIg1lgmABY4PiA29m7UQYihHhEgAI6GXEA
\begin{tikzcd}
Y \arrow[r, "f"] \arrow[rd, "p"'] & Y' \arrow[d, "q"] \\
                                  & X                
\end{tikzcd}
\end{center}

\end{Def}
La categoría $\Top /X$ es un ejemplo de ``categoría coma" (ver por ejemplo \cite[p.~45]{CWM} donde se denota por $(\Top \downarrow X)$). La notación $\Bund (X)$ es debida a \textit{bundle}, la traducción al inglés de la palabra \textit{manojo}. Un concepto esencial al trabajar con manojos es el de \textit{sección}:
\begin{Def}[Secciones]
   Sea $p:Y\to X$ un manojo sobre $X$.
   \begin{itemize}
      \item Una flecha $s:1_{X}\to p$ en $\Top/X$ es llamada una sección transversal de $p$. En este caso decimos que $s$ es una sección global de $p$.
      \item Dado $U\subseteqab X$, una flecha $s:in_{U,X}\to p$ en $\Top/X$ es llamada una sección transversal de $p$ sobre $U$. En este caso decimos que $s$ es una sección local de $p$.
   \end{itemize}
\end{Def}
De esta forma, una sección global (local) es una función continua que hace que el siguiente diagrama de la izquierda (derecha) conmute:
\begin{center}
    % https://tikzcd.yichuanshen.de/#N4Igdg9gJgpgziAXAbVABwnAlgFyxMJZABgBoBGAXVJADcBDAGwFcYkQANEAX1PU1z5CKchWp0mrdl179seAkVHFxDFm0QgAmjz4gM8oUQBMYmmqmaAqrrmDFKAMxmJ66bf0CFw5M5XnJDW0ecRgoAHN4IlAAMwAnCABbJFMQHAgkUVdLfQ94pKQyNIzELIsg8gB9LhpGegAjGEYABS8jTTiscIALHDyE5MQi9JSAt00EWRB8wedipABWMZzJvRnFmhHEABZloLR+gsQ5rd20+ixGdm6ICABrEFqGptbDBxBOnr699iwwSuAVlIHG4IW4QA
\begin{tikzcd}
                                    & Y \arrow[d, "p"] &                                                  & Y \arrow[d, "p"] \\
X \arrow[r, "1_X"'] \arrow[ru, "s"] & X                & U \arrow[ru, "s"] \arrow[r, "{in_{U,X}}"', hook] & X               
\end{tikzcd}
\end{center}

Vemos también que $s$ se presenta como una inversa (local), a derecha, de $p$.
