Es la acción de ir a $\PreSh(X)$ por $\Gamma$ y volver a $\Top/X$ por $\Lambda$ lo que genera la transformación natural $\epsilon$. Deseamos que $\epsilon$ sea una transformación natural entre los funtores $\Lambda \Gamma$ y $1_{\Top/X}$ de $\Top/X$ en $\Top/X$. Dado un manojo $p:Y\to X$ sobre $X$, debemos definir un manojo $\epsilon_Y:\Lambda \Gamma Y \to Y$. Notemos que
$$
   \Lambda \Gamma Y = \Lambda_{\Gamma Y} = \bigcup \mathcal{B}_{\Gamma Y} = \bigcup_{\substack{U\in\mathcal{O}(X)\\s\in(\Gamma Y)U}}\dot{s}(U)
$$
de modo que un elemento de $\Lambda \Gamma Y$ es de la forma $\dot{s}(x)$ para algunos $U\in\mathcal{O}(X)$, $x\in U$ y $s\in(\Lambda Y)(U)$. Observando esto es natural, para cada manojo $p:Y\to X$, definir $\epsilon Y:\Lambda \Gamma Y\to 1_{\Top/X}$ mediante $\epsilon Y (\dot{s}(x))=s(x)\in Y$. La representación de $\dot{s}(x)$ en $\Lambda \Gamma Y$ puede no ser única, esto es, pueden existir $V\in \mathcal{O}(X)$ y $t\in(\Gamma Y)V$ tales que $\dot{s}(x)=\dot{t}(x)$, es decir, $s$ y $t$ tienen el mismo germen en $x$, pero esto implica que en alguna vecindad de $x$, $s=t$, y en particular $s(x)=t(x)$ con lo cual $\epsilon_{Y}$ está bien definida. Igualmente, puede probarse que $\epsilon_Y$ es continua, para posteriormente ver que es una función entre manojos sobre $X$.  Esta asignación de $\epsilon_{Y}$ para cada $Y\in\Top/X$ hace de $\epsilon$ una transformación natural $\Lambda \Gamma \dot{\to} 1_{\Top/X}$. Además, cuando $Y$ es un espacio Étalé, puede probarse, dando una inversa explícita, que $\epsilon_Y:\Lambda \Gamma Y \to Y$ es un isomorfismo. En particular, para cualquier $P\in \PreSh(X)$, sabemos que $\Lambda P$ es Étalé sobre $X$, y por tanto $\epsilon_{\Lambda P}:\Lambda \Gamma \Lambda P \to \Lambda P$ es un isomorfismo; esto nos da que $\Lambda$ y $\Gamma$ satisfacen la propiedad $\textbf{(L2)}$ del Lema \ref{Lema:LemaAdjuntos}.

Ahora bien, es fácil ver que los siguientes diagramas (en ${\PreSh(X)}^{\Top/X}$ a la izquierda, ${\Top/X}^{\PreSh(X)}$ a la derecha) son conmutativos:
% https://tikzcd.yichuanshen.de/#N4Igdg9gJgpgziAXAbVABwnAlgFyxMJZABgBpiBdUkANwEMAbAVxiRAB12BxOgW17ogAvqXSZc+QigCM5KrUYs2nHvzoACTgBk+AIygaVfAcNEgM2PASIAmOdXrNWiDt2OCRYy5KIBmewpOyuw6vPoeZhYS1igALAGOSi7aegaabmrpoeGmXtFSyACsCYrOrtkGwvIwUADm8ESgAGYAThC8SGQgOBBIsoFJrjA4dEZquSCt7X3UPUh2A2VjApwwaNgM1p6TbR2IXXOIC7owYFBIALSxAJwOpWzSAPrAy3RCINQMdCcMAAriVikIBaWFqAAscBMpnt-N1eoh4otghVDOxhhFmrskIjDsUkck0essJswFlUhidtNELDcdQTmdLjc7kEXE8XiFye9Pt8YH8AT4XCDwZChBQhEA
\begin{center}
\begin{tikzcd}
\Gamma \arrow[r, "\eta\Gamma"] \arrow[rr, "1_{\Gamma}"', bend right=49] & \Gamma \Lambda \Gamma \arrow[r, "\Gamma\epsilon"] & \Gamma & \Lambda \arrow[r, "\Lambda \eta"] \arrow[rr, "1_{\Lambda}"', bend right=49] & \Lambda \Gamma \Lambda \arrow[r, "\epsilon \Lambda"] & \Lambda
\end{tikzcd}
\end{center}

basta seguir los diagramas conmutativos que surgen gracias a las correspondientes transformaciones naturales. Por lo tanto, del  Teorema \ref{Tma:TmaAdjuntos1} se sigue que $\Lambda$ y $\Gamma$ son funtores adjuntos.

Ahora bien, sabemos que $\Sh(X)$ y $\Etale(X)$ son subcategorías plenas de $\PreSh(X)$ y $\Top/X$; además, en la Sección \ref{section:BundToPreSh} probamos que cada manojo $\left( Y,p\right)$ sobre $X$ determina un haz (de secciones transversales) $\Gamma Y$ (es decir $\Gamma Y\in \Sh(X)$); igualmente en la Sección \ref{section:PreShToBund} probamos que cada prehaz $P$ sobre $X$ determina un espacio Étalé $\Lambda P$ sobre $X$ (es decir, $\Lambda_P\in\Etale(X))$. Por tanto, el Lema \ref{Lema:LemaAdjuntos} es aplicable a las categorías $\Sh(X)$ y $\Etale(X)$ en función de $\mathcal{P}_0$ y $\mathcal{B}_0$, respectivamente. Con éste podemos concluir que las categorías $\Sh(X)$ y $\Etale(X)$ son equivalentes para cada espacio topológico $X$. 
